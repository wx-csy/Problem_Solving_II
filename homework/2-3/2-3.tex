\documentclass[a4paper,11pt]{article}
\usepackage{ctex}
\usepackage{enumerate}
\usepackage{times}
\usepackage{mathptmx}
\usepackage{amsmath}
\usepackage{amsfonts}
\usepackage{amssymb}
\usepackage{clrscode3e}
\usepackage{tikz}
\usepackage[top=2cm, bottom=2cm, left=2cm, right=2cm]{geometry}

%\allowdisplaybreaks[4]
\renewcommand{\labelenumi}{\textbf{\emph{\alph{enumi}}.}}
\begin{document}
  \title{����~2-3~��ҵ}
  \author{��������ۿԴ \and ѧ�ţ�161240004}
  \date{}
  \maketitle

  \section{[TC] Problem 4.1-5}
  \begin{codebox}
    \Procname{$\proc{Find-Maximum-Subarray}(A, n)$}
    \li $ans = 0$
    \li $sum = 0$
    \li \For $i = 1$ \To $n$
    \li \Do
            $sum = sum + A[i]$
    \li     \If $sum > ans$
    \li     \Then
                $ans = sum$
    \li     \ElseIf $sum < 0$
            \Then
    \li         $sum = 0$
            \End
        \End
    \li \Return $ans$
  \end{codebox}

  \section{[TC] Problem 4.3-3}
  We have to prove that $T(n) \geq cn \lg n$ for appropriate choice of constant $c > 0$.  We use mathematical induction to prove it. \par
  For the base step, we have $T(1)=1$, $T(2) = 4$ and $T(3) = 5$. We should choose $c$ such that $T(1)=1 \geq 0$, $T(2) = 4 \geq 2c$, $T(3) = 5 \geq 3c \lg 3$. We can choose every $c < 1$. \par
  For the induction step, assume that $T(m) \geq cm \lg m$ holds for every positive integer $m < n$, where $n \geq 4$. Substituting it into the recurrence, we obtain
  \begin{align*}
    T(n) &= 2 T(\lfloor n/2 \rfloor) + n \\
         &\geq 2 (c \lfloor n/2 \rfloor \lg \lfloor n/2 \rfloor) + n \\
         &\geq 2 (c (n/2-1) \lg (n/2-1)) + n \\
         &\geq c(n-2)\lg((n-2)/2)+n \\
         &= cn\lg(n-2)-2c\lg(n-2)+cn-2c+n\\
         &= cn\lg n - cn\lg\frac{n}{n-2}-2c\lg(n-2)+cn-2c+n\\
         &= cn\lg n + (cn- cn\lg\frac{n}{n-2}) + (n/2-2c\lg(n-2))+(n/2-2c)\\
         &\geq cn\lg n
  \end{align*}
  the last step holds if we choose $c < \frac{1}{2}$, because
  \begin{enumerate}[(1)]
    \item $n \geq 4 \Rightarrow \lg\frac{n}{n-2} \leq 1 \Rightarrow cn \geq cn\lg\frac{n}{n-2}$ ;
    \item $n \geq 4 \Rightarrow n/2 \geq \lg(n-2)$, $(n/2 \geq \lg(n-2)) \wedge (c < \frac{1}{2}) \Rightarrow n/2<2c\lg(n-2)$;
    \item $n \geq 4 \Rightarrow n/2 > 1$, $(n/2 \geq 1) \wedge (c < \frac{1}{2}) \Rightarrow n/2 > 2c$.
  \end{enumerate} \par
  By mathematical induction, we conclude that for all positive constant $c < \frac{1}{2}$, $T(n) \geq cn\lg n$ holds. Hence the recurrence is also $\Omega(n \lg n)$. Therefore we conclude that the solution is $\Theta(n \lg n)$.

  \section{[TC] Problem 4.3-7}
  Assume $T(m) \leq c m^{\log_3 4}$ for sufficiently large $m < n$, especially for $m = n/3$ where $n$ is large enough. Substituting into the recurrence yields
  \begin{align*}
    T(n) &= 4 T(n/3) + n \\
         &\leq 4c(\frac{n}{3})^{\log_3 4} + n \\
         &= cn^{\log_3 4} + n \\
         &\nleq cn^{\log_3 4} & fails!
  \end{align*}

  We guess $T(n) \leq c(n^{\log_3 4}-n)$ instead and prove it by mathematical induction. \par
  For the base step, when $n=3$, we have $T(n) = 7$ and $c(n^{\log_3 4}-n) = c$, therefore $T(n) \leq c(n^{\log_3 4}-n)$ holds for every $c > 7$. \par
  For the induction step, assume $T(m) \leq c(m^{\log_3 4}-m)$ holds for every $m < n$, where $n > 3$. Substituting into the recurrence yields
    \begin{align*}
    T(n) &= 4 T(n/3) + n \\
         &\leq 4c((\frac{n}{3})^{\log_3 4}-\frac{n}{3}) + n \\
         &= cn^{\log_3 4} + (1-\frac{4c}{3}) n \\
         &\leq cn^{\log_3 4}
    \end{align*} \par
  By mathematical induction, we conclude $T(n) \leq c(n^{\log_3 4}-n)$. Therefore $T(n) = O(n^{\log_3 4}-n) = O(n^{\log_3 4})$. \par
  To complete the proof of $T(n) = \Omega(n^{\log_3 4})$, we assume $T(n) \geq c n^{\log_3 4}$. \par
  For the base step, when $n=1$, we have $T(n) = 1$ and $c n^{\log_3 4} = c$, therefore $T(n) \geq c n^{\log_3 4}$ holds for every positive constant $c < 1$. \par
  For the induction step, assume $T(m) \geq c m^{\log_3 4}$ holds for every $m < n$, where $n > 3$. Substituting into the recurrence yields
    \begin{align*}
    T(n) &= 4 T(n/3) + n \\
         &\geq 4c(\frac{n}{3})^{\log_3 4} + n \\
         &= cn^{\log_3 4} + n \\
         &\geq cn^{\log_3 4}
    \end{align*} \par
  By mathematical induction, we conclude that $T(n) \geq c n^{\log_3 4}$. Hence $T(n) = \Omega(n^{\log_3 4})$. Therefore $T(n) = \Theta(n^{\log_3 4})$.

  \section{[TC] Problem 4.4-2}
  Recursion tree:\\
  \begin{tikzpicture}
    \node (l11) at (6,0) {$n^2$};
    \node (l21) at (4,-1.5) {$(\frac{n}{2})^2$};
    \node (l22) at (8,-1.5) {$(\frac{n}{2})^2$};
    \draw[-] (l11) -- (l21);
    \draw[-] (l11) -- (l22);
    \node (l31) at (3,-3) {$(\frac{n}{4})^2$};
    \node (l32) at (5,-3) {$(\frac{n}{4})^2$};
    \node (l33) at (7,-3) {$(\frac{n}{4})^2$};
    \node (l34) at (9,-3) {$(\frac{n}{4})^2$};
    \draw[-] (l21) -- (l31);
    \draw[-] (l21) -- (l32);
    \draw[-] (l22) -- (l33);
    \draw[-] (l22) -- (l34);
    \draw[-, dashed] (l31) -- (2.5,-4);
    \draw[-, dashed] (l31) -- (3.5,-4);
    \draw[-, dashed] (l32) -- (4.5,-4);
    \draw[-, dashed] (l32) -- (5.5,-4);
    \draw[-, dashed] (l33) -- (6.5,-4);
    \draw[-, dashed] (l33) -- (7.5,-4);
    \draw[-, dashed] (l34) -- (8.5,-4);
    \draw[-, dashed] (l34) -- (9.5,-4);
    \node at (6, -4.5) {$\vdots$};
    \node (t1) at (11, 0) {$n^2$};
    \node (t2) at (11, -1.5) {$\frac{n^2}{2}$};
    \node (t3) at (11, -3) {$\frac{n^2}{4}$};
    \draw[->, dashed] (l11) -- (t1);
    \draw[->, dashed] (l22) -- (t2);
    \draw[->, dashed] (l34) -- (t3);
    \node (atop) at (1, 0.4) {};
    \node (nl) at (1,-2.2) {$\lg n$};
    \node (abottom) at (1, -5) {};
    \draw[->, thick] (nl) -- (atop);
    \draw[->, thick] (nl) -- (abottom);
    \node at (11, -4.5) {$\vdots$};
    \draw[-] (9.5, -5) -- (11.2, -5);
    \node at (10.3, -5.5) {Total: $O(n^2)$};
  \end{tikzpicture}

  We guess that $T(n) \leq cn^2$ for an appropriate choice of positive constant $c$. For the base step, we have $T(1)=1$ and $cn^2=c$, so $T(n) \leq cn^2$ holds for $c>1$ when $n=1$. \par
  For the induction step, assume that $T(m) \leq cm^2$ holds for all positive integer $m<n$, where $n>1$. Substituting into the recurrence yields
  \begin{align*}
    T(n) &= T(n/2) + n^2 \\
         &\leq c^2\frac{n^2}{4} + n^2 \\
         &= (1+\frac{c^2}{4})n^2 \\
         &= cn^{2}
  \end{align*}
  where the last step holds as long as $c=2$. \par
  By mathematical induction, we conclude that $T(n) \leq 2n^2$, i.e. $T(n) = O(n^2)$.

  \section{[TC] Problem 4.4-8}
  Recursion tree: \\
  \begin{tikzpicture}
    \node (l11) at (6,0) {$cn$};
    \node (l21) at (4,-1.5) {$c(n-a)$};
    \node (l22) at (8,-1.5) {$T(a)$};
    \draw[-] (l11) -- (l21);
    \draw[-] (l11) -- (l22);
    \node (l31) at (3,-3) {$c(n-2a)$};
    \node (l32) at (5,-3) {$T(a)$};
    \draw[-] (l21) -- (l31);
    \draw[-] (l21) -- (l32);
    \draw[-, dashed] (l31) -- (2.5,-4);
    \draw[-, dashed] (l31) -- (3.5,-4);
    \node at (6, -4.5) {$\vdots$};
    \node (t1) at (11, 0) {$cn$};
    \node (t2) at (11, -1.5) {$c(n-a)+T(a)$};
    \node (t3) at (11, -3) {$c(n-2a)+T(a)$};
    \draw[->, dashed] (l11) -- (t1);
    \draw[->, dashed] (l22) -- (t2);
    \draw[->, dashed] (l32) -- (t3);
    \node (atop) at (1, 0.4) {};
    \node (nl) at (1,-2.2) {$\lg n$};
    \node (abottom) at (1, -5) {};
    \draw[->, thick] (nl) -- (atop);
    \draw[->, thick] (nl) -- (abottom);
    \node at (11, -4.5) {$\vdots$};
    \draw[-] (9.5, -5) -- (11.2, -5);
    \node at (10.3, -5.5) {Total: $O(n^2)$};
  \end{tikzpicture}
  \par
  Adding up all levels of the tree, we get
  \begin{align*}
    T(n) &= cn + \sum_{i=1}^{n/a} (c(n-ia)+T(a)) \\
         &= \frac{n}{a}T(a)+(\frac{n}{a}+1)cn-ca(1+\frac{n}{a})(\frac{n}{a})/2 \\
         &= \frac{T(a)}{a}n+cn-cn/2+\frac{c}{a}n^2-\frac{c}{2a}n^2\\
         &= \left(\frac{T(a)}{a}+\frac{c}{2}\right)n+\frac{c}{2a}n^2\\
         &= O(n^2)
  \end{align*}

  \section{[TC] Problem 4-1}
  \begin{enumerate}
  \item $T(n) = O(n^4) = \Omega(n^4)$ \par
    For this recurrence, we have $a=2$, $b=2$, $f(n)=n^4$, and thus we have $n^{\log_a b}=n^{\log_2 2}=n$. Hence $f(n) = \Omega(n^{\log_2 2 + \epsilon})$ where $\epsilon = 3$. The regularity condition holds for $f(n)$ because $2f(n/2)=2(n/2)^4=n^4/8 \leq cn^4$ where $c = 1/8 < 1$. Therefore, the master theorem, case 3 applies and we obtain $T(n) = \Theta(n^4)$.
  \item $T(n) = O(n) = \Omega(n)$ \par
    For this recurrence, we have $a=1$, $b=10/7$, $f(n)=n$, and thus we have $n^{\log_a b}=n^{\log_{10/7} 1}=n^0$. Hence $f(n) = \Omega(n^{\log_{10/7} 1 + \epsilon})$ where $\epsilon = 1$. The regularity condition holds for $f(n)$ because $f(7n/10)=7n/10\leq cn$ where $c=7/10<1$. Therefore, the master theorem, case 3 applies and we obtain $T(n) = \Theta(n)$.
  \item $T(n) = O(n^2\lg n) = \Omega(n^2\lg n)$ \par
    For this recurrence, we have $a=16$, $b=4$, $f(n)=n^2$, and thus we have $n^{\log_a b}=n^{\log_4 16}=n^2$. Hence $f(n)=\Theta(n^2)$. Therefore, the master theorem, case 2 applies and we obtain $T(n)=\Theta(n^2\lg n)$.
  \item $T(n)=O(n^2)=\Omega(n^2)$ \par
    For this recurrence, we have $a=7$, $b=3$, $f(n)=n^2$, and thus we have $n^{\log_a b}=n^{\log_3 7}$. Hence $f(n)=\Omega(n^{\log_3 7 + \epsilon})$ where $\epsilon = 2 - \log_3 7 > 0$. The regularity condition holds for $f(n)$ because $7f(n/3) = 7(n/3)^2=7n/9=cn$ where $c=7/9<1$. Therefore, the master theorem, case 3 applies and we obtain $T(n) = \Theta(n^2)$.
  \item $T(n)=O(n^{\log_2 7})=\Omega(n^{\log_2 7})$ \par
    For this recurrence, we have $a=7$, $b=2$, $f(n)=n^2$, and thus we have $n^{\log_a b}=n^{\log_2 7}$. Hence $f(n)=O(n^{\log_2 7 - \epsilon})$ where $\epsilon = \log_2 7-2 >0$. Therefore, the master theorem, case 1 applies and we obtain $T(n) = \Theta(n^{\log_2 7})$.
  \item $T(n)=O(\sqrt{n}\lg n)=\Omega(\sqrt{n}\lg n)$ \par
    For this recurrence, we have $a=2$, $b=4$, $f(n)=\sqrt{n}$, and thus we have $n^{\log_a b}=n^{\log_4 2}=\sqrt{n}$. Hence $f(n)=\Theta(\sqrt{n})$. Therefore, the master theorem, case 2 applies and we obtain $T(n)=\Theta(\sqrt(n) \lg n)$.
  \item $T(n)=O(n^3)=\Omega(n^3)$. \par
    We use mathematical induction to prove the upper bounds and the lower bounds. \par
    First, we have to prove $T(n) = O(n^3)$, i.e, $T(n) \leq cn^3$ for an appropriate choice of positive constant $c$.\par
    For the base step, $T(1) \leq c$, $T(2) \leq 8c$, $T(3) \leq 27c$, $T(4) \leq 64c$, $T(5) = 125c$, so $T(n) \leq cn^3$ holds for all $c > \max \{ T(1), T(2)/8, T(3)/27, T(4)/64, T(5)/125\}$ when $n=1,2,3,4,5$. \par
    For the induction step, assume $T(m) \leq cm^3$ holds for all $m<n$ where $n>5$. Substituting into the recurrence yields
    \begin{align*}
      T(n) &= T(n-2) + n^2 \\
           &\leq c(n-2)^3 + n^2 \\
           &= cn^3+(1-6c)n^2 + 24cn -8c \\
           &\leq cn^3+((1-6c)n+24c)n \\
           &\leq cn^3
    \end{align*}
    the last step holds if we choose $c>1$ because $n>5$. \par
    By mathematical induction, we conclude that $T(n) \leq cn^3$ holds for all positive integer $n$ when $c > \max\{1, T(1), T(2)/8, T(3)/27, T(4)/64, T(5)/125\}$, i.e. $T(n) = O(n^3)$. \par
    Second, we have to prove that $f(n) \geq cn^3$ for an appropriate choice of positive constant $c$. \par
    For the base step, $T(1) \geq c$, $T(2) \geq 8c$, so $T(n) \geq cn^3$ holds for all $c < \min \{ T(1), T(2)/8 \}$ when $n=1,2$. \par
    For the induction step, assume $T(m) \geq cm^3$ holds for all $m<n$ where $n>2$. Substituting into the recurrence yields
    \begin{align*}
      T(n) &= T(n-2) + n^2 \\
           &\geq c(n-2)^3+n^2 \\
           &=cn^3 + (1-6c)n^2 + 24cn -8c \\
           &\geq cn^3
    \end{align*}
    the last step holds if we choose $c<1/6$. \par
    By mathematical induction, we conclude that $T(n) \geq cn^3$ holds for all positive integer $n$ when $c < \min \{1/6, T(1), T(2)/8 \} $, i.e. $T(n) = \Omega(n^3)$.
  \end{enumerate}
\end{document}
