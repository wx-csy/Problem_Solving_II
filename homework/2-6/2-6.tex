\documentclass[a4paper,11pt]{article}
\usepackage{ctex}
\usepackage{enumerate}
\usepackage{times}
\usepackage{mathptmx}
\usepackage{amsmath}
\usepackage{amssymb}
\usepackage{clrscode3e}
\usepackage[top=2cm, bottom=2cm, left=2cm, right=2cm]{geometry}

\allowdisplaybreaks[4]
\renewcommand{\labelenumi}{\textbf{\emph{\alph{enumi}}.}}
\begin{document}
  \title{����~2-6~��ҵ}
  \author{��������ۿԴ \and ѧ�ţ�161240004}
  \date{}
  \maketitle

  \section{[CS] Problem 5.6-4}
  Let $X$ denote the amount of money one wins by playing this game once, and $X$ is a random variable. The expectation of $X$ satisfies
  $$E(X) = \frac{1}{4}(1+E(X)) + \frac{1}{4} \times 2 + \frac{1}{4} \times 3 + \frac{1}{4} \times 4$$
  Solve this equation, we obtain
  $$E(X) = \frac{10}{3} \approx 3.33$$ \par
  Therefore, the maximum amount of money a rational person would pay to play this game is \$3.33 .

  \section{[CS] Problem 5.6-8}
  \begin{align*}
    \sum_{i=1}^{n} E(X|F_i)P(F_i) &= \sum_{i=1}^n P(F_i) \sum_{s:s \in F_i} X(s) \frac{P(s)}{P(F_i)} \\
    &= \sum_{i=1}^n \sum_{s:s \in F_i} X(s)P(s) \\
    &= \sum_{s:s \in S} X(s)P(s) \\
    &= E(X)
  \end{align*}

  \section{[CS] Problem 5.7-1}
  $X$ follows the binomial distribution with parameters $n=5$ and $p=0.6$. The expectation and variance is
  $$E(X) = 5 \times 0.6 = 3$$
  $$V(X) = 5 \times 0.6 \times 0.4 = 1.2$$
  Therefore,
  $$E(X-3) = E(X) - E(3) = 3 - 3 =0$$
  $$E((X-3)^2) = E((X-E(X))^2) = V(X) = 1.2$$

  \section{[CS] Problem 5.7-2}
  Every question is one Bernoulli trial with probability 0.6 of success, therefore
  $E(X_i) = p = 0.6$ and $V(X_1) = p(1-p) = 0.24$. 
  The sum of the variances of $X_1$ through $X_5$ is 1.2, which equals to the variance of $X$, because random variables $X_1$ through $X_5$ are independent.

  \section{[CS] Problem 5.7-4}
  Let random variable $X$ be the number of right answers. $X$ follows the binomial distribution with parameters $n=100$ and $p=0.6$. Therefore
  $$ E(X) = np = 100 \times 0.6 = 60$$
  $$ V(X) = np(1-p) = 100 \times 0.6 \times 0.4 = 24 $$
  $$ \sigma(X) = \sqrt{V(X)} = 2 \sqrt{6} \approx 4.90 $$

  \section{[CS] Problem 5.7-6}
  Let random variable $X_i$ be the number of right answers in an $i$-question quiz. $X_i$ follows the binomial distribution with parameters $n=i$ and $p=0.8$. Therefore
  $$ V(X_{25}) = 25p(1-p) = 4$$
  $$ V(X_{100}) = 100p(1-p) = 16$$
  $$ V(X_{400}) = 400p(1-p) = 64$$ \par
  To ``correct'' these variances, we can use the standard deviation, i.e. the square root of the variance, instead of variance.

  \section{[CS] Problem 5.7-12}
  Assume there are $n$ questions on a short-answer test. Let random variable $X$ denote the number of questions a student who knows 80\% of the course material answers correctly. $X$ follows the binomial distribution with parameter $n$ and $p=0.8$. \par
  By the central limit theorem, the distribution of $X$ converges to the normal distribution with expectation $np$ and variance $np(1-p)$, as $n$ grows large. Hence, if we are 95\% sure that such student gets a grade between 75\% and 85\%, about 2 standard deviations should not be greater than 5\% of $n$, i.e.
  $$ 2\sqrt{np(1-p)} \leq 0.05n $$
  Solve this inequality and we get
  $$ n \geq 1600 p(1-p) = 256$$ \par
  Therefore, about 256 questions are needed.

  \section{[CS] Problem 5.7-16}
  \begin{enumerate}
    \item
    \begin{align*}
      V(X) &= E((X-E(X))^2) \\
       &= \sum_{i=1}^n(X(x_i)-E(X))^2P(x_i) \\
       &\geq \sum_{i=1}^k (X(x_i)-E(X))^2P(x_i) \\
       &> \sum_{i=1}^k P(x_i) r^2 \\
       &= P(E) r^2
    \end{align*}
    \item
    Dividing by $r^2$ on both sides yields
    $$ P(E) < V(X)/r^2 $$
    That means, the probability of $|X(x) - E(X)| \geq r$ is no more than $V(X)/r^2$.
  \end{enumerate}

  \section{[CS] Problem 5.7-18}
  \begin{enumerate}
    \item $X$ follows the binomial distribution with parameter $n$ and $p$. The expectation of $X$ is $np$. By Chebyshev's law, we get
    $$ P(|X(x)-np| \geq sn) = P(|X(x)-E(X)| \geq sn) \leq V(X)/(s^2n^2) = np(1-p)/(s^2n^2) = p(1-p)/(s^2n) $$

    \item For every positive $\epsilon$ (arbitrarily small), if we take $n > \frac{p(1-p)}{s^2 \epsilon}$, $P(|X(x) - np| < ns) > 1 - \epsilon $ holds, because
        \begin{align*}
          P(|X(x) - np| < ns) &= 1 - P(|X(x)-E(X)| \geq sn) \\
          & > 1 - p(1-p)/(s^2n) \\
          &\geq 1 - \frac{p(1-p)}{s^2} \frac{s^2\epsilon}{p(1-p)} \\
          &= 1 - \epsilon
        \end{align*}
    That means, the probability of $X(x)$ being between $np-sn$ and $np+sn$ can be arbitrarily close to 1 if $n$ is sufficiently large.
  \end{enumerate}

  \section{[TC] Problem 5.2-1}
  When hiring exactly once, the first candidate is the best candidate among the $n$ candidates. Therefore, the probability of hiring exactly once is $1/n$. \par
  When hiring exactly $n$ times, the candidates are in increasing order of quality. Since there are $n!$ possible permutations of these candidates, the probability of hiring exactly $n$ times is $1/n!$.

  \section{[TC] Problem 5.2-2}
  The first candidate must be hired. Despite the first candidate, exactly one of the remaining candidates is hired. Let $p_i$ denote the probability that the $i-$th candidate is hired, and we have $p_i = 1/i$. Then the probability of hiring twice, denoted as $P$, is
  \begin{align*}
    P &= \sum_{i=2}^{n} (1-p_2) \cdots p_i \cdots (1-p_n) \\
    &= \sum_{i=2}^{n} (1-p_2) \cdots (1-p_n) \frac{p_i}{1-p_i}\\
    &= \sum_{i=2}^{n} \left( \prod_{j=2}^{n} (1-p_j) \right) \frac{p_i}{1-p_i}\\
    &= \sum_{i=2}^{n} \left( \prod_{j=2}^{n} \frac{j-1}{j} \right) \frac{1/i}{1-1/i}\\
    &= \frac{1}{n} \sum_{i=2}^{n} \frac{1}{i-1}\\
    &= \frac{\ln n}{n} + \Theta(1/n)
  \end{align*}

  \section{[TC] Problem 5.2-4}
  Let $X_i$ be the indicator random variable associated with the event that the $i-$th customer gets his own hat back, the probability of which is $1/n$. We have $X = X_1 + X_2 + \cdots + X_n$, and by the linearity of expectation, we get
  $$ E[X] = E \left[\sum_{i=1}^{n} X_i\right] = \sum_{i=1}^{n} E[X_i] = \sum_{i=1}^{n} 1/n = 1$$

  \section{[TC] Problem 5.3-1}
  \begin{codebox}
    \Procname{$\proc{Randomize-In-Place}(A)$}
    \li $n = A.length$
    \li swap $A[1]$ with $A[\proc{Random}(1, n)]$
    \li \For $i = 2$ \To $n$
    \li \Do swap $A[i]$ with $A[\proc{Random}(i, n)]$
        \End
  \end{codebox} \par
  The loop invariant, the \textbf{Maintenance} part and the \textbf{Termination} part of the modified proof are just the same as the original version. The only difference is the \textbf{Initialization} part. \par
  \textbf{Initialization:} Consider the situation just before the first loop iteration, so that $i=2$. The loop invariant says that for each possible 1-permutation, the subarray contains this 1-permutation with probability $(n-i+1)!/n! = (n-1)!/n! = 1/n$. The subarray $A[1..1]$ is an array contains only one element, which is randomly picked from the original array in line 2. Therefore, $A[1..1]$ contains any 1-permutation with probability $1/n$, and the loop invariant holds prior to the first iteration.

  \section{[TC] Problem 5.3-2}
  No. He wants to produce a non-identity permutation randomly, i.e. every possible non-identity permutation can be produced with the same probability. However, because the procedure swaps the first element with one of the rest elements, it cannot produce permutations such that the $A[1]$ is the original element, but $A[2..n]$ contains a non-identity permutation of the rest elements.

  \section{[TC] Problem 5.3-3}
  When $n \leq 2$, this code obviously produces a uniform random permutation. However, when $n>2$, it doesn't. \par
  In each iteration the procedure swaps the current element with a random element in the whole array. Therefore, there are $n^n$ ways to swap the elements, each having the probability $1/n^n$. However, there are $n!$ possible permutations in total. We define a equivalence relation on these $n^n$ ways, that two ways of swapping are equivalent if and only if the two ways produces the same permutation. If this procedure produces a uniform random permutation, the equivalence relation divides the $n^n$ ways into $n!$ equivalence classes of the same size. Hence, the size of each equivalent class is $n^n/n!$. However, when $n>2$, $n^n/n!$ is not an integer, because $(n-1)$ is a factor of the denominator but not a factor of the numerator when $n>2$.

  \section{[TC] Problem 5.3-4}
  $A[i]$ winds up in position $j$ in $B$ when $\emph{offset} \mod n = (j-i) \mod n \in \{0,1, \cdots ,n-1\}$, where $\emph{offset}$ is uniformly distributed from 1 to $n$, i.e $\emph{offset} \mod n$ is uniformly distributed from 0 to $n-1$. Therefore, each element $A[i]$ has a $1/n$ probability of winding up in any particular position in $B$. \par
  This procedure performs a right circular shift by $\emph{offset} \mod n$ positions on the original array, where $\emph{offset} \mod n$ is a random integer uniformly distributed from 0 to $n-1$. Therefore, the procedure can only produce $n$ of the $n!$ possible permutations of the original array, i.e. the resulting permutation is not uniformly random.
  
  \section{[TC] Problem 5.2}
  \begin{enumerate}
    \item Pseudocode:
    \begin{codebox}
      \Procname{$\proc{Randomize-Search}(A, x)$}
      \li $n = A.length$
      \li $r = n$
      \li let $T$ be a new array of Boolean
      \li \For $i = 1$ \To $n$
      \li \Do $T[i] = false$
          \End
      \li \While $r \neq 0$
          \Do 
      \li $t = \proc{Random}(1,n)$
      \li \If $A[t] = x$
          \Do
      \li \Return $t$
      \li \Else
      \li \If $T[t] = false$
          \Do
      \li $T[t] = true$
      \li $r = r - 1$
          \End
          \End
          \End
      \zi \Comment $x$ is not in the array
      \li \Return $-1$
    \end{codebox}
    \item Assume $X$ is the number of indices into $A$ that we pick before we find $x$. Then we have
    $$ E(X) = \frac{1}{n} \times 1 + \frac{n-1}{n} (E(X) + 1)$$
    Solve this equation and we get
    $$ E(X) = n $$
    \item Assume $X$ is the number of indices into $A$ that we pick before we find $x$. Then we have
    $$ E(X) = \frac{k}{n} \times 1 + \frac{n-k}{n} (E(X) + 1)$$
    Solve this equation and we get
    $$ E(X) = n/k $$
    \item Let $X_i$ denote the number of indices into $A$ that we pick before we find an element that has not checked before, when $i$ elements in $A$ has been checked. We have
        $$ E(X_i) = \frac{n-i}{n} \times 1 + \frac{i}{n} (E(X_i) + 1)$$
        Solve this equation, we get
        $$ E(X_i) = \frac{n}{n-i} $$
        Let $X$ denote the number of indices into $A$ that we must pick before we have checked all elements of $A$ and \proc{Random-Search} terminates. We have $X = \sum_{i=0}^{n-1} X_i$, and by the linearity of expectation, we get
        \begin{align*}
          E[X] &= E\left[\sum_{i=0}^{n-1} X_i\right] \\
          &= \sum_{i=0}^{n-1} E[X_i] \\
          &= \sum_{i=0}^{n-1} \frac{n}{n-i} \\
          &= n \sum_{i=1}^{n} \frac{1}{i} \\
          &= n \ln n + \Theta(n)
        \end{align*}
  \end{enumerate}
\end{document}
